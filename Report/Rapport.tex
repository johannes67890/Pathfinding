% !TeX root = Rapport.tex

\documentclass[12pt]{article}
\usepackage{lingmacros}
\usepackage{tree-dvips}
\usepackage[utf8]{inputenc}
\usepackage{fancyhdr}
\usepackage{listings}
\usepackage{xcolor}
\usepackage{geometry}
 \geometry{
 a4paper,
 total={170mm,257mm},
 left=15mm,
 top=20mm,
 }


\definecolor{comment}{rgb}{0,0.45,0}
\definecolor{codegray}{rgb}{0.5,0.5,0.5}
\definecolor{codepurple}{rgb}{0.58,0,0.82}
\definecolor{backcolour}{rgb}{0.95,0.95,0.92}
\lstdefinestyle{CodeStyle}{
    backgroundcolor=\color{backcolour},   
    commentstyle=\color{comment},
    keywordstyle=\color{magenta},
    numberstyle=\tiny\color{codegray},
    stringstyle=\color{codepurple},
    basicstyle=\ttfamily\footnotesize,
    breakatwhitespace=false,         
    breaklines=true,                 
    captionpos=b,                    
    keepspaces=true,                 
    numbers=left,                    
    numbersep=5pt,                  
    showspaces=false,                
    showstringspaces=false,
    showtabs=false,                  
    tabsize=2
}
\lstset{style=CodeStyle}
\lstdefinelanguage{JavaScript}{
  keywords={typeof, new, true, false, catch, function, return, null, catch, switch, var, if, in, while, do, else, case, break},
  keywordstyle=\color{purple}\bfseries,
  ndkeywords={class, export, const, var, let, boolean, throw, implements, import, this, !!, !=, ===, ;},
  ndkeywordstyle=\color{blue}\bfseries,
  identifierstyle=\color{black},
  sensitive=false,
  comment=[l]{//},
  morecomment=[s]{/*}{*/},
  commentstyle=\color{comment}\ttfamily,
  stringstyle=\color{orange}\ttfamily,
  morestring=[b]',
  morestring=[b]"
}

\begin{document}

\title{Dijkstra's Algoritme}
\author{Johannes Jørgensen\\ S2o}
\date{2021 Febuar}
\maketitle
\pagebreak
\tableofcontents
\pagebreak

\section{Introduktion}
Dijkstra's Algoritme er en kendt algoritmen indenfor pathfinding. Formål er at finde den korteste vej fra punkt a til punkt b. Dette er en større del af almindelige menneskers liv end man tænker. Hvad er den hurtigste vej til skole eller arbejde? Er det via motorvejen som måske har vejarbejde? Vil det så være hurtigere at tag vejen igennem byen? Disse spørgsmål kan man ofte få hurtigt svar på med ens GPS, som har implementeret diverse pathfinding algoritmer og er tilkoblet internettet med de seneste nyheder om vejarbejde, kø osv. 
\\Jeg vil med brug af Dijkstra's Algoritme finde sammenhængen mellem den matematiske del af rekursion som og et rekursion-kald i programmering.  

\section{Teori}
\subsection{Hvad er rekursion?}
Ordet rekursion er en betegnelse for noget som referere til sig selv. Når noget referere sig selv betyder det at noget behøver information eller data fra sit forrige jeg. Rekursion er oftest betegnet som en rekursionligning i matematik og en som en funktion der kalder sig selv i programmering. 
\subsection{Rekursion i Matematik}
\subsection{Programmering}
\subsubsection{Rekursion i programmering}
Rekursion i programmering er en funktion som har et rekursivts kald, alså funktion kalder sig selv. Dette kald kan være meget direkte som i eksempel 1 (Simpel rekursions funktion), eller indirekte hvor det ikke er lige så overskuligt at se et rekursivt kald.
\begin{lstlisting}[language=JavaScript, caption=Simpel rekursions funktion]
function fractorial(n) {        
  if (n == 1) return 1; 
  else return n*fractorial(n-1) 
} // if n = 4, then exprected output: 24
\end{lstlisting}
Funktionen i eksempel 1 indtag en værdi \textit{n} som parameter. Hvis denne parameteren \textit{n} er 1 stopper det rekursive kald (se linje 2.). Dette betyder også at dette eksempel har et meget direkte grænse hvor \textit{n} ikke kan komme under 1. Linje 3 beskriver så hvordan \textit{n} skal reagere hvis dens værdi ikke er 1. Hvis \textit{n} ikke er 1, men eksempelvis 3 skal den gange nuværende \textit{n} med forrige \textit{n}. I forrige \textit{n} af 3 er 2, derfor skal 2 ganges med dens forrige \textit{n} og således.
\[n(1) = 1 \Leftrightarrow 1\] 
\[n(2) = 2*1 \Leftrightarrow   2\]
\[n(3) = 3*2 \Leftrightarrow   6\]
\[n(4) = 4*6 \Leftrightarrow  \textbf{24}\]


\subsubsection{Variabler}
Et variabel er en pladsholder for et stykke data. Typisk set bruger man variabler i brug når man skal gemme et stykke data som man skal bruge forskellige steder i programmet. Disse variabler kan ændres under programmets køretid afhængig i hvordan variables data bliver manipuleret. Variablers data kan eventuelt ændres under programmets køretid, hvis man ønsker. Lidt mere teknisk indeholder et variable særlige set af bits eller af variablenes datatype. Et variable har en datatype som identificere hvilke slags data variabelt kan indeholde. Diverse programmeringssprog bruger dynamiske datatyper, hvor variabler kan indeholde forskellige datatyper. Her et en tabel med forskellige datatyper.
\begin{table}[ht]
  \centering
  \begin{tabular}{ |c|c|c|c| }
   \hline
   \textbf{Datatype} & \textbf{Skrivemåder} & \textbf{Eksempel på data} & \textbf{Kommentar} \\ 
   \hline
   Integer & int & \small{…-3, -2, -1, 0, 1, 2,... 1000} & \shortstack{Kun heltal\\ \footnotesize{(både negativ og positiv)}} \\
   \hline
   Floating Point & float & …, -3.25, -2.11,... 100,12  & \shortstack{Decimaltal\\ \footnotesize{(både negativ og positiv)}} \\ 
   \hline
   String & str eller text & \shortstack{"hello world”, \\ "This is a string"}  &  \shortstack{Et række af karakterer\\ oftest tekst.} \\ 
   \hline
   Array & [ data1,data2,… ] & int val = [-2,-1,0,1,5] & \shortstack{Indeholder en række data \\ under et navn} \\ 
   \hline
   Character & char & @, s, ¤, k…  & Kun et symbol \\ 
   \hline
   Boolean & bool & True (1) eller False (0)  & Enten sandt eller falsk \\ 
   \hline
   Constant & const & "Hello", -3, 22, [1, 2, 3] & \shortstack{Dynamisk datatype, \\ men konstant værdi} \\ 
   \hline
  \end{tabular}
\end{table}
\subsubsection{Objekter}
\subsubsection{Arrays}
Et array er en datastruktur, der indeholder en gruppe af elementer. Disse elementer er typisk alle af samme datatype, såsom et integer eller en string. Arrays bruges typisk set i computerprogrammer til at organisere data, så et relateret sæt værdier nemt kan sorteres eller søges.
\begin{lstlisting}[language=JavaScript, caption=Eksempel på et array]
// Array with constant values of datatype String
const cars = ["Audi", "Volvo", "BMW"];
cars[0] = "Audi" // Index 0 of array
\end{lstlisting}
\subsubsection{Løkker}
En løkke er et stykke kode som vil fortsætte med at køre indtil en given betingelse er opfyldt. der er to slags løkker, “while” - løkker og “for”-løkker. “While” løkker kører ind til betingelsen i parentesen bliver opfyldt. Mens “for” løkker lavet selv et variabel, som i det simpleste tilfælde vil tælle op eller ned fra og køre det antal gange ifølge betingelse. Løkker er meget brugbare siden det stopper en fra at skulle skrive mange linjer af den samme kode hvis man skal køre noget flere gange.
\begin{lstlisting}[language=JavaScript, caption=Eksempel på et for løkke]
for (let i = 0; i < 5; i++) {
  console.log(i);
} // exprected output: 0,1,2,3,4
\end{lstlisting}
\subsubsection{Betinget udførelse (if statement)}
Betinget udførelse eller “if statement”, er et stykke kode som checker om en betingelse opfyldt nogle specifikke kriterier. Hvis de betingelser bliver opfyldt vil koden inde i dette if statement blive kørt. Med et if statement kan man give den flere kriterier ved at lave en “else if” eller “else” statement efter den første stykke kode. “else if” er et til if statement med en betingelse før den bliver kørt, samtidig med at dette stykke kode kun bliver kørt hvis det første if statement ikke opfylder sin betingelse. Et “else” statement er bare et stykke kode som vil blive kørt hvis koden ikke opfyldte sin betingelse.
\begin{lstlisting}[language=JavaScript, caption=Eksempel på betinget udførelse]
let val = 5; 

if(val == 1){
  console.log("The value is now 1!");
}else console.log("The value is not 1"); // exprected execution of statement
\end{lstlisting}
% dijkstra algo
\begin{lstlisting}[language=JavaScript, caption=Dijkstra's Algoritme]
function dijkstra(
  grid: CellProps[][],
  startNode: CellProps,
  finishNode: CellProps
) {
  const visitedNodesInOrder = [];
  startNode.distance = 0;
  const unvisitedNodes = getAllCells(grid);
  while (!!unvisitedNodes.length) {
    sortCellsByDistance(unvisitedNodes);
    const closestNode: CellProps | undefined = unvisitedNodes.shift();

    if (closestNode != undefined) {
      // If we encounter a wall, we skip it.
      if (closestNode.isWall) continue;
      // If the closest node is at a distance of infinity,
      // we must be trapped and should therefore stop.
      if (closestNode.distance === Infinity) return visitedNodesInOrder;
      closestNode.isVisited = true;
      visitedNodesInOrder.push(closestNode);
      if (closestNode === finishNode) return visitedNodesInOrder;
      updateUnvisitedNeighbors(closestNode, grid);
    } else console.log("error, closestNode returned 0");
  }
}
\end{lstlisting}

\end{document}